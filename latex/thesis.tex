%# -*- coding: utf-8-unix -*-
%%==================================================
%% thesis.tex
%%==================================================

% 双面打印
\documentclass[doctor, openright, twoside]{sjtuthesis}
% \documentclass[bachelor, openany, oneside, submit]{sjtuthesis}
% \documentclass[master, review]{sjtuthesis}
% \documentclass[%
%   bachelor|master|doctor,	% 必选项
%   fontset=fandol|windows|mac|ubuntu|adobe|founder, % 字体选项
%   oneside|twoside,		% 单面打印,双面打印(奇偶页交换页边距,默认)
%   openany|openright, 		% 可以在奇数或者偶数页开新章|只在奇数页开新章(默认)
%   english,			% 启用英文模版
%   review,	 		% 盲审论文,隐去作者姓名、学号、导师姓名、致谢、发表论文和参与的项目
%   submit			% 定稿提交的论文,插入签名扫描版的原创性声明、授权声明 
% ]

% 逐个导入参考文献数据库
\addbibresource{bib/thesis.bib}
% \addbibresource{bib/chap2.bib}

\begin{document}

%% 无编号内容:中英文论文封面、授权页
$if(title)$
\title{$title$}
$endif$
$if(author)$
\author{$author$}
$endif$
$if(advisor)$
\advisor{$advisor$}
$endif$
$if(coadvisor)$
\coadvisor{$coadvisor$}
$endif$
$if(defenddate)$
\defenddate{$defenddate$}
$endif$
$if(school)$
\school{$school$}
$endif$
$if(institute)$
\institute{$institute$}
$endif$
$if(studentnumber)$
\studentnumber{$studentnumber$}
$endif$
$if(major)$
\major{$major$}
$endif$

$if(englishtitle)$
\englishtitle{$englishtitle$}
$endif$
$if(englishauthor)$
\englishauthor{\textsc{$englishauthor$}}
$endif$
$if(englishadvisor)$
\englishadvisor{Prof. \textsc{$englishadvisor$}}
$endif$
$if(englishcoadvisor)$
\englishcoadvisor{Prof. \textsc{$englishcoadvisor$}}
$endif$
$if(englishschool)$
\englishschool{$englishschool$}
$endif$
$if(englishinstitute)$
\englishinstitute{\textsc{$englishinstitute$} \\
  \textsc{Shanghai Jiao Tong University} \\
  \textsc{Shanghai, P.R.China}}
$endif$
$if(englishmajor)$
\englishmajor{$englishmajor$}
$endif$
$if(englishdate)$
\englishdate{$englishdate$}
$endif$
\maketitle

\makeatletter
\ifsjtu@submit\relax
	\includepdf{pdf/original.pdf}
	\cleardoublepage
	\includepdf{pdf/authorization.pdf}
	\cleardoublepage
\else
\ifsjtu@review\relax
% exclude the original claim and authorization
\else
	\makeDeclareOriginal
	\makeDeclareAuthorization
\fi
\fi
\makeatother


\frontmatter 	% 使用罗马数字对前言编号

%% 摘要
\pagestyle{main}
\include{tex/abstract}

%% 目录、插图目录、表格目录
\tableofcontents
\listoffigures
\addcontentsline{toc}{chapter}{\listfigurename} %将插图目录加入全文目录
\listoftables
\addcontentsline{toc}{chapter}{\listtablename}  %将表格目录加入全文目录
\listofalgorithms
\addcontentsline{toc}{chapter}{\listalgorithmname} %将算法目录加入全文目录

\include{tex/symbol} % 主要符号、缩略词对照表

\mainmatter	% 使用阿拉伯数字对正文编号

%% 正文内容
\pagestyle{main}
\include{tex/intro}
\include{tex/example}
\include{tex/faq}
\include{tex/summary}

\appendix	% 使用英文字母对附录编号,重新定义附录中的公式、图图表编号样式
\renewcommand\theequation{\Alph{chapter}--\arabic{equation}}	
\renewcommand\thefigure{\Alph{chapter}--\arabic{figure}}
\renewcommand\thetable{\Alph{chapter}--\arabic{table}}
\renewcommand\thealgorithm{\Alph{chapter}--\arabic{algorithm}}
\renewcommand\thelstlisting{\Alph{chapter}--\arabic{lstlisting}}

%% 附录内容,本科学位论文可以用翻译的文献替代。
\include{tex/app_setup}
\include{tex/app_eq}
\include{tex/app_cjk}
\include{tex/app_log}

\backmatter	% 文后无编号部分 

%% 参考资料
\printbibliography[heading=bibintoc]

%% 致谢、发表论文、申请专利、参与项目、简历
%% 用于盲审的论文需隐去致谢、发表论文、申请专利、参与的项目
\makeatletter

%%
% "研究生学位论文送盲审印刷格式的统一要求"
% http://www.gs.sjtu.edu.cn/inform/3/2015/20151120_123928_738.htm

% 盲审删去删去致谢页
\ifsjtu@review\relax\else
  \include{tex/ack} 	  %% 致谢
\fi

\ifsjtu@bachelor
  % 学士学位论文要求在最后有一个英文大摘要,单独编页码
  \pagestyle{biglast}
  \include{tex/end_english_abstract}
\else
  % 盲审论文中,发表学术论文及参与科研情况等仅以第几作者注明即可,不要出现作者或他人姓名
  \ifsjtu@review\relax
    \include{tex/pubreview}
    \include{tex/projectsreview}  
  \else
    \include{tex/pub}	      %% 发表论文
    \include{tex/projects}  %% 参与的项目
  \fi
\fi

% \include{tex/patents}	  %% 申请专利
% \include{tex/resume}	  %% 个人简历

\makeatother

\end{document}
